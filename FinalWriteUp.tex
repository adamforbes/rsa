\documentclass[12pt]{article}

\usepackage{amsmath}
\usepackage{amssymb}
\usepackage{amsfonts}

\setcounter{secnumdepth}{2}
\setcounter{tocdepth}{2}


\usepackage[margin=1in]{geometry}

\usepackage{fancyhdr}
\pagestyle{fancy}
\lhead{}
\chead{}
\rhead{COMP 360: Matt Adelman, Evan Carmi, Adam Forbes}
\lfoot{}
\cfoot{\thepage}
\rfoot{}

\newcommand{\HRule}{\rule{\linewidth}{0.5mm}}

\begin{document}

\begin{titlepage}

\begin{center}
% Upper part of the page
\textsc{\LARGE COMP 360}\\[1.5cm]
\textsc{\Large Final Project}\\[0.5cm]
% Title
\HRule \\[0.4cm]
{ \huge \bfseries RSA Moduli Pitfalls}\\[0.4cm]
\HRule \\[1.5cm]
% Author and supervisor
\begin{minipage}{0.4\textwidth}
\begin{center} \large
\emph{Authors:}\\
Matt Adelman\\Evan Carmi\\Adam Forbes
\end{center}
\end{minipage}
\vfill
% Bottom of the page
{\large \today}
\end{center}
\end{titlepage}

\tableofcontents

\newpage

\begin{@twocolumnfalse}

\begin{abstract}
  Hello world, this is the abstract.
\end{abstract}
\end{@twocolumnfalse}


\section{History:}
\indent\indent
RSA is a public key encryption algorithm developed by Ron Rivest, Adi Shamir, 
and Leonard Adleman at MIT in 1977. Rivest and Shamir are both computer
scientists that were working on an ``unbreakable'' public key encryption method.
Rivest and Shamir worked on many different codes, and would pose them as a
challenge to Adleman. Forty two of these codes were presented, and Adleman broke
them all. Finally on attempty number 43, they created what is now known as the
RSA scheme. Incidentally, English Professor CLifford Cocks developed the exact
same encryption system in 1973, but it was classified as top-secret, so it was
not released until 1997. The RSA algorithm was released for public domain in
1997.\\\indent
The algorithm operates by using two distinct, large prime numbers to generate
public and private keys. Anyone can use the public key to encrypt a message, but
only someone with the private key can decrypt it. The algorithm is hard to
break, because if the prime numbers are large enough, the factorization is
exponential in time.

\section{Original Ideas:}

\section{Algorithm Implementation:}
For 

\section{Data Collection:}

\section{Analysis:}

\section{Problems:}

\section{Conclusion:}

\section{Further Research:}

\appendix
\section{Appendix of code:}


\end{document}

Abstract (All):

History (Matt):

Lit Review (Adam):
 - RWWWIR

Original ideas:

Algorithm Implementaion (Matt):
 - What we did
 - Big int package used
 - Why web implementaion

Data collection (Evan):
- SSL observatoru 2010
- new data collection
- parsig new data
- analysing new data (and old data)

analysis

conclusion, things to do better

further research/work
   - database of moduli that you can check on

- code



Possible tables and graphs
- top ten shared moduli
- number of new data collected
    - distribution of key size
-Time complexity of expected vs. actual
