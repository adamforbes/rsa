\documentclass[11pt,twoside]{article}


% ------
% Fonts and typesetting settings
\usepackage[sc]{mathpazo}
\usepackage[T1]{fontenc}
\linespread{1.05} % Palatino needs more space between lines
\usepackage{microtype}

\usepackage{hyperref}
\usepackage{textcomp}

% ------
% Page layout
\usepackage[hmarginratio=1:1,top=32mm,columnsep=30pt,margin=1in]{geometry}
\usepackage[font=it]{caption}
\usepackage{paralist}
\usepackage{multicol}

% ------
% Lettrines
\usepackage{lettrine}


% ------
% Abstract
\usepackage{abstract}
	\renewcommand{\abstractnamefont}{\normalfont\bfseries}
	\renewcommand{\abstracttextfont}{\normalfont\small}

\usepackage{titlesec}
\titleformat{\section}[block]{\large\scshape\centering}{\thesection.}{1em}{}

\usepackage{amsmath}
\usepackage{amssymb}
\usepackage{amsfonts}
\usepackage{listings}

% listings.tex - A LaTeX file that configures some settings for the listings package.
% Author: Evan Carmi http://ecarmi.org/
% Based off configurations from Professor Norman Danner (Wesleyan University).
%
% Usage: to include this file in another .tex file add:
%   % listings.tex - A LaTeX file that configures some settings for the listings package.
% Author: Evan Carmi http://ecarmi.org/
% Based off configurations from Professor Norman Danner (Wesleyan University).
%
% Usage: to include this file in another .tex file add:
%   % listings.tex - A LaTeX file that configures some settings for the listings package.
% Author: Evan Carmi http://ecarmi.org/
% Based off configurations from Professor Norman Danner (Wesleyan University).
%
% Usage: to include this file in another .tex file add:
%   \input{/path/to/listings.tex}
%   and the style for the code like:
%   \lstset{style=pseudocode}
% to the preamble of that document.
%
% Example:
%--------------------
%   \input{/home/evan/tex/listings.tex}
%   \lstset{style=pseudocode}
%
%   \begin{document}
%
%   \begin{lstlisting}
%   r = 0
%   i = n*n
%   while i > 1 do
%       for j = 1 to i do
%           r = r + 1
%       i = i/2
%   \end{lstlisting}
%--------------------
%End Example

\usepackage{listings}
\lstdefinestyle{ccode}{
	language=C,
	keywordstyle=\ttfamily\underbar,
	morekeywords={DEFINE},
	keywordstyle=[2]\texttt,
	morekeywords=[2]{GLfloat},
	identifierstyle=\texttt,
	columns=flexible,
}
\lstdefinestyle{pythoncode}{
		language=Python,
        keywordstyle=\ttfamily\underbar,
        keywordstyle=[2]\texttt,
        identifierstyle=\texttt,
		commentstyle=\texttt,
		stringstyle=\texttt,
		tabsize=4,
		showstringspaces=false,
}
\lstdefinestyle{javacode}{
		language=Java,
        keywordstyle=\ttfamily\underbar,
		morekeywords={enum,assert},
        keywordstyle=[2]\texttt,
        identifierstyle=\texttt,
		commentstyle=\texttt,
		stringstyle=\texttt,
		tabsize=4,
		showstringspaces=false,
}
\lstdefinestyle{pseudocode}{
		language=pseudo,
        keywordstyle=\normalfont\textbf,
        keywordstyle=[2]\normalfont\textsf,
		keywordstyle=[3]\normalfont\textsf,
        identifierstyle=\normalfont\textit,
		commentstyle=\textrm,
		stringstyle=\texttt,
		tabsize=4,
		showstringspaces=false,
}
\lstdefinelanguage{pseudo}%
  {%identifierstyle=\normalfont\textit,
   columns=fullflexible,
   mathescape=true,
%   numbers=left,
%   numberblanklines=false,
   breaklines=true,
   breakatwhitespace=true,
   morekeywords={if,then,else,elseif,endif,switch,endswitch,case,while,do,endw,repeat,until,for,foreach,in,endf,break,begin,end,function,return,or,and,to,by,input,output,continue,type,of},
   morekeywords=[2]{int, float, double, bool, char, string, seq, list, array, set, map, void, stack, collection},
   morekeywords=[3]{true, false, null},
   % keywordstyle=\normalfont\textbf,
   literate={<-}{{${}\leftarrow{}$}}{2} {[]}{[ ]}{2} {!=}{{${}\not={}$}}{1} {<=}{${}\leq{}$}{1} {>=}{${}\geq{}$}{1},
   % keywordstyle=[2]\normalfont\texttt,
   morecomment=[l]{Input:},
   morecomment=[l]{Parameter},
   morecomment=[l]{Parameters:},
   morecomment=[l]{Returns:},
   morecomment=[l]{Effect:},
   morecomment=[l]{Pre-conditions:},
   morecomment=[l]{State:},
   morecomment=[l]{Messages:},
   morecomment=[l]{//},
   morecomment=[s]{(*}{*)},
   moredelim=[is]{(**}{**)},
   morestring=[b]",
   % morestring=[b]',
  }
  [keywords]

% \def\keyw#1{\lstinline!#1!}
% \def\jkeyw#1{\lstinline[style=javacode]!#1!}
% \def\pskeyw#1{\lstinline[style=pseudocode]!#1!}

%   and the style for the code like:
%   \lstset{style=pseudocode}
% to the preamble of that document.
%
% Example:
%--------------------
%   % listings.tex - A LaTeX file that configures some settings for the listings package.
% Author: Evan Carmi http://ecarmi.org/
% Based off configurations from Professor Norman Danner (Wesleyan University).
%
% Usage: to include this file in another .tex file add:
%   \input{/path/to/listings.tex}
%   and the style for the code like:
%   \lstset{style=pseudocode}
% to the preamble of that document.
%
% Example:
%--------------------
%   \input{/home/evan/tex/listings.tex}
%   \lstset{style=pseudocode}
%
%   \begin{document}
%
%   \begin{lstlisting}
%   r = 0
%   i = n*n
%   while i > 1 do
%       for j = 1 to i do
%           r = r + 1
%       i = i/2
%   \end{lstlisting}
%--------------------
%End Example

\usepackage{listings}
\lstdefinestyle{ccode}{
	language=C,
	keywordstyle=\ttfamily\underbar,
	morekeywords={DEFINE},
	keywordstyle=[2]\texttt,
	morekeywords=[2]{GLfloat},
	identifierstyle=\texttt,
	columns=flexible,
}
\lstdefinestyle{pythoncode}{
		language=Python,
        keywordstyle=\ttfamily\underbar,
        keywordstyle=[2]\texttt,
        identifierstyle=\texttt,
		commentstyle=\texttt,
		stringstyle=\texttt,
		tabsize=4,
		showstringspaces=false,
}
\lstdefinestyle{javacode}{
		language=Java,
        keywordstyle=\ttfamily\underbar,
		morekeywords={enum,assert},
        keywordstyle=[2]\texttt,
        identifierstyle=\texttt,
		commentstyle=\texttt,
		stringstyle=\texttt,
		tabsize=4,
		showstringspaces=false,
}
\lstdefinestyle{pseudocode}{
		language=pseudo,
        keywordstyle=\normalfont\textbf,
        keywordstyle=[2]\normalfont\textsf,
		keywordstyle=[3]\normalfont\textsf,
        identifierstyle=\normalfont\textit,
		commentstyle=\textrm,
		stringstyle=\texttt,
		tabsize=4,
		showstringspaces=false,
}
\lstdefinelanguage{pseudo}%
  {%identifierstyle=\normalfont\textit,
   columns=fullflexible,
   mathescape=true,
%   numbers=left,
%   numberblanklines=false,
   breaklines=true,
   breakatwhitespace=true,
   morekeywords={if,then,else,elseif,endif,switch,endswitch,case,while,do,endw,repeat,until,for,foreach,in,endf,break,begin,end,function,return,or,and,to,by,input,output,continue,type,of},
   morekeywords=[2]{int, float, double, bool, char, string, seq, list, array, set, map, void, stack, collection},
   morekeywords=[3]{true, false, null},
   % keywordstyle=\normalfont\textbf,
   literate={<-}{{${}\leftarrow{}$}}{2} {[]}{[ ]}{2} {!=}{{${}\not={}$}}{1} {<=}{${}\leq{}$}{1} {>=}{${}\geq{}$}{1},
   % keywordstyle=[2]\normalfont\texttt,
   morecomment=[l]{Input:},
   morecomment=[l]{Parameter},
   morecomment=[l]{Parameters:},
   morecomment=[l]{Returns:},
   morecomment=[l]{Effect:},
   morecomment=[l]{Pre-conditions:},
   morecomment=[l]{State:},
   morecomment=[l]{Messages:},
   morecomment=[l]{//},
   morecomment=[s]{(*}{*)},
   moredelim=[is]{(**}{**)},
   morestring=[b]",
   % morestring=[b]',
  }
  [keywords]

% \def\keyw#1{\lstinline!#1!}
% \def\jkeyw#1{\lstinline[style=javacode]!#1!}
% \def\pskeyw#1{\lstinline[style=pseudocode]!#1!}

%   \lstset{style=pseudocode}
%
%   \begin{document}
%
%   \begin{lstlisting}
%   r = 0
%   i = n*n
%   while i > 1 do
%       for j = 1 to i do
%           r = r + 1
%       i = i/2
%   \end{lstlisting}
%--------------------
%End Example

\usepackage{listings}
\lstdefinestyle{ccode}{
	language=C,
	keywordstyle=\ttfamily\underbar,
	morekeywords={DEFINE},
	keywordstyle=[2]\texttt,
	morekeywords=[2]{GLfloat},
	identifierstyle=\texttt,
	columns=flexible,
}
\lstdefinestyle{pythoncode}{
		language=Python,
        keywordstyle=\ttfamily\underbar,
        keywordstyle=[2]\texttt,
        identifierstyle=\texttt,
		commentstyle=\texttt,
		stringstyle=\texttt,
		tabsize=4,
		showstringspaces=false,
}
\lstdefinestyle{javacode}{
		language=Java,
        keywordstyle=\ttfamily\underbar,
		morekeywords={enum,assert},
        keywordstyle=[2]\texttt,
        identifierstyle=\texttt,
		commentstyle=\texttt,
		stringstyle=\texttt,
		tabsize=4,
		showstringspaces=false,
}
\lstdefinestyle{pseudocode}{
		language=pseudo,
        keywordstyle=\normalfont\textbf,
        keywordstyle=[2]\normalfont\textsf,
		keywordstyle=[3]\normalfont\textsf,
        identifierstyle=\normalfont\textit,
		commentstyle=\textrm,
		stringstyle=\texttt,
		tabsize=4,
		showstringspaces=false,
}
\lstdefinelanguage{pseudo}%
  {%identifierstyle=\normalfont\textit,
   columns=fullflexible,
   mathescape=true,
%   numbers=left,
%   numberblanklines=false,
   breaklines=true,
   breakatwhitespace=true,
   morekeywords={if,then,else,elseif,endif,switch,endswitch,case,while,do,endw,repeat,until,for,foreach,in,endf,break,begin,end,function,return,or,and,to,by,input,output,continue,type,of},
   morekeywords=[2]{int, float, double, bool, char, string, seq, list, array, set, map, void, stack, collection},
   morekeywords=[3]{true, false, null},
   % keywordstyle=\normalfont\textbf,
   literate={<-}{{${}\leftarrow{}$}}{2} {[]}{[ ]}{2} {!=}{{${}\not={}$}}{1} {<=}{${}\leq{}$}{1} {>=}{${}\geq{}$}{1},
   % keywordstyle=[2]\normalfont\texttt,
   morecomment=[l]{Input:},
   morecomment=[l]{Parameter},
   morecomment=[l]{Parameters:},
   morecomment=[l]{Returns:},
   morecomment=[l]{Effect:},
   morecomment=[l]{Pre-conditions:},
   morecomment=[l]{State:},
   morecomment=[l]{Messages:},
   morecomment=[l]{//},
   morecomment=[s]{(*}{*)},
   moredelim=[is]{(**}{**)},
   morestring=[b]",
   % morestring=[b]',
  }
  [keywords]

% \def\keyw#1{\lstinline!#1!}
% \def\jkeyw#1{\lstinline[style=javacode]!#1!}
% \def\pskeyw#1{\lstinline[style=pseudocode]!#1!}

%   and the style for the code like:
%   \lstset{style=pseudocode}
% to the preamble of that document.
%
% Example:
%--------------------
%   % listings.tex - A LaTeX file that configures some settings for the listings package.
% Author: Evan Carmi http://ecarmi.org/
% Based off configurations from Professor Norman Danner (Wesleyan University).
%
% Usage: to include this file in another .tex file add:
%   % listings.tex - A LaTeX file that configures some settings for the listings package.
% Author: Evan Carmi http://ecarmi.org/
% Based off configurations from Professor Norman Danner (Wesleyan University).
%
% Usage: to include this file in another .tex file add:
%   \input{/path/to/listings.tex}
%   and the style for the code like:
%   \lstset{style=pseudocode}
% to the preamble of that document.
%
% Example:
%--------------------
%   \input{/home/evan/tex/listings.tex}
%   \lstset{style=pseudocode}
%
%   \begin{document}
%
%   \begin{lstlisting}
%   r = 0
%   i = n*n
%   while i > 1 do
%       for j = 1 to i do
%           r = r + 1
%       i = i/2
%   \end{lstlisting}
%--------------------
%End Example

\usepackage{listings}
\lstdefinestyle{ccode}{
	language=C,
	keywordstyle=\ttfamily\underbar,
	morekeywords={DEFINE},
	keywordstyle=[2]\texttt,
	morekeywords=[2]{GLfloat},
	identifierstyle=\texttt,
	columns=flexible,
}
\lstdefinestyle{pythoncode}{
		language=Python,
        keywordstyle=\ttfamily\underbar,
        keywordstyle=[2]\texttt,
        identifierstyle=\texttt,
		commentstyle=\texttt,
		stringstyle=\texttt,
		tabsize=4,
		showstringspaces=false,
}
\lstdefinestyle{javacode}{
		language=Java,
        keywordstyle=\ttfamily\underbar,
		morekeywords={enum,assert},
        keywordstyle=[2]\texttt,
        identifierstyle=\texttt,
		commentstyle=\texttt,
		stringstyle=\texttt,
		tabsize=4,
		showstringspaces=false,
}
\lstdefinestyle{pseudocode}{
		language=pseudo,
        keywordstyle=\normalfont\textbf,
        keywordstyle=[2]\normalfont\textsf,
		keywordstyle=[3]\normalfont\textsf,
        identifierstyle=\normalfont\textit,
		commentstyle=\textrm,
		stringstyle=\texttt,
		tabsize=4,
		showstringspaces=false,
}
\lstdefinelanguage{pseudo}%
  {%identifierstyle=\normalfont\textit,
   columns=fullflexible,
   mathescape=true,
%   numbers=left,
%   numberblanklines=false,
   breaklines=true,
   breakatwhitespace=true,
   morekeywords={if,then,else,elseif,endif,switch,endswitch,case,while,do,endw,repeat,until,for,foreach,in,endf,break,begin,end,function,return,or,and,to,by,input,output,continue,type,of},
   morekeywords=[2]{int, float, double, bool, char, string, seq, list, array, set, map, void, stack, collection},
   morekeywords=[3]{true, false, null},
   % keywordstyle=\normalfont\textbf,
   literate={<-}{{${}\leftarrow{}$}}{2} {[]}{[ ]}{2} {!=}{{${}\not={}$}}{1} {<=}{${}\leq{}$}{1} {>=}{${}\geq{}$}{1},
   % keywordstyle=[2]\normalfont\texttt,
   morecomment=[l]{Input:},
   morecomment=[l]{Parameter},
   morecomment=[l]{Parameters:},
   morecomment=[l]{Returns:},
   morecomment=[l]{Effect:},
   morecomment=[l]{Pre-conditions:},
   morecomment=[l]{State:},
   morecomment=[l]{Messages:},
   morecomment=[l]{//},
   morecomment=[s]{(*}{*)},
   moredelim=[is]{(**}{**)},
   morestring=[b]",
   % morestring=[b]',
  }
  [keywords]

% \def\keyw#1{\lstinline!#1!}
% \def\jkeyw#1{\lstinline[style=javacode]!#1!}
% \def\pskeyw#1{\lstinline[style=pseudocode]!#1!}

%   and the style for the code like:
%   \lstset{style=pseudocode}
% to the preamble of that document.
%
% Example:
%--------------------
%   % listings.tex - A LaTeX file that configures some settings for the listings package.
% Author: Evan Carmi http://ecarmi.org/
% Based off configurations from Professor Norman Danner (Wesleyan University).
%
% Usage: to include this file in another .tex file add:
%   \input{/path/to/listings.tex}
%   and the style for the code like:
%   \lstset{style=pseudocode}
% to the preamble of that document.
%
% Example:
%--------------------
%   \input{/home/evan/tex/listings.tex}
%   \lstset{style=pseudocode}
%
%   \begin{document}
%
%   \begin{lstlisting}
%   r = 0
%   i = n*n
%   while i > 1 do
%       for j = 1 to i do
%           r = r + 1
%       i = i/2
%   \end{lstlisting}
%--------------------
%End Example

\usepackage{listings}
\lstdefinestyle{ccode}{
	language=C,
	keywordstyle=\ttfamily\underbar,
	morekeywords={DEFINE},
	keywordstyle=[2]\texttt,
	morekeywords=[2]{GLfloat},
	identifierstyle=\texttt,
	columns=flexible,
}
\lstdefinestyle{pythoncode}{
		language=Python,
        keywordstyle=\ttfamily\underbar,
        keywordstyle=[2]\texttt,
        identifierstyle=\texttt,
		commentstyle=\texttt,
		stringstyle=\texttt,
		tabsize=4,
		showstringspaces=false,
}
\lstdefinestyle{javacode}{
		language=Java,
        keywordstyle=\ttfamily\underbar,
		morekeywords={enum,assert},
        keywordstyle=[2]\texttt,
        identifierstyle=\texttt,
		commentstyle=\texttt,
		stringstyle=\texttt,
		tabsize=4,
		showstringspaces=false,
}
\lstdefinestyle{pseudocode}{
		language=pseudo,
        keywordstyle=\normalfont\textbf,
        keywordstyle=[2]\normalfont\textsf,
		keywordstyle=[3]\normalfont\textsf,
        identifierstyle=\normalfont\textit,
		commentstyle=\textrm,
		stringstyle=\texttt,
		tabsize=4,
		showstringspaces=false,
}
\lstdefinelanguage{pseudo}%
  {%identifierstyle=\normalfont\textit,
   columns=fullflexible,
   mathescape=true,
%   numbers=left,
%   numberblanklines=false,
   breaklines=true,
   breakatwhitespace=true,
   morekeywords={if,then,else,elseif,endif,switch,endswitch,case,while,do,endw,repeat,until,for,foreach,in,endf,break,begin,end,function,return,or,and,to,by,input,output,continue,type,of},
   morekeywords=[2]{int, float, double, bool, char, string, seq, list, array, set, map, void, stack, collection},
   morekeywords=[3]{true, false, null},
   % keywordstyle=\normalfont\textbf,
   literate={<-}{{${}\leftarrow{}$}}{2} {[]}{[ ]}{2} {!=}{{${}\not={}$}}{1} {<=}{${}\leq{}$}{1} {>=}{${}\geq{}$}{1},
   % keywordstyle=[2]\normalfont\texttt,
   morecomment=[l]{Input:},
   morecomment=[l]{Parameter},
   morecomment=[l]{Parameters:},
   morecomment=[l]{Returns:},
   morecomment=[l]{Effect:},
   morecomment=[l]{Pre-conditions:},
   morecomment=[l]{State:},
   morecomment=[l]{Messages:},
   morecomment=[l]{//},
   morecomment=[s]{(*}{*)},
   moredelim=[is]{(**}{**)},
   morestring=[b]",
   % morestring=[b]',
  }
  [keywords]

% \def\keyw#1{\lstinline!#1!}
% \def\jkeyw#1{\lstinline[style=javacode]!#1!}
% \def\pskeyw#1{\lstinline[style=pseudocode]!#1!}

%   \lstset{style=pseudocode}
%
%   \begin{document}
%
%   \begin{lstlisting}
%   r = 0
%   i = n*n
%   while i > 1 do
%       for j = 1 to i do
%           r = r + 1
%       i = i/2
%   \end{lstlisting}
%--------------------
%End Example

\usepackage{listings}
\lstdefinestyle{ccode}{
	language=C,
	keywordstyle=\ttfamily\underbar,
	morekeywords={DEFINE},
	keywordstyle=[2]\texttt,
	morekeywords=[2]{GLfloat},
	identifierstyle=\texttt,
	columns=flexible,
}
\lstdefinestyle{pythoncode}{
		language=Python,
        keywordstyle=\ttfamily\underbar,
        keywordstyle=[2]\texttt,
        identifierstyle=\texttt,
		commentstyle=\texttt,
		stringstyle=\texttt,
		tabsize=4,
		showstringspaces=false,
}
\lstdefinestyle{javacode}{
		language=Java,
        keywordstyle=\ttfamily\underbar,
		morekeywords={enum,assert},
        keywordstyle=[2]\texttt,
        identifierstyle=\texttt,
		commentstyle=\texttt,
		stringstyle=\texttt,
		tabsize=4,
		showstringspaces=false,
}
\lstdefinestyle{pseudocode}{
		language=pseudo,
        keywordstyle=\normalfont\textbf,
        keywordstyle=[2]\normalfont\textsf,
		keywordstyle=[3]\normalfont\textsf,
        identifierstyle=\normalfont\textit,
		commentstyle=\textrm,
		stringstyle=\texttt,
		tabsize=4,
		showstringspaces=false,
}
\lstdefinelanguage{pseudo}%
  {%identifierstyle=\normalfont\textit,
   columns=fullflexible,
   mathescape=true,
%   numbers=left,
%   numberblanklines=false,
   breaklines=true,
   breakatwhitespace=true,
   morekeywords={if,then,else,elseif,endif,switch,endswitch,case,while,do,endw,repeat,until,for,foreach,in,endf,break,begin,end,function,return,or,and,to,by,input,output,continue,type,of},
   morekeywords=[2]{int, float, double, bool, char, string, seq, list, array, set, map, void, stack, collection},
   morekeywords=[3]{true, false, null},
   % keywordstyle=\normalfont\textbf,
   literate={<-}{{${}\leftarrow{}$}}{2} {[]}{[ ]}{2} {!=}{{${}\not={}$}}{1} {<=}{${}\leq{}$}{1} {>=}{${}\geq{}$}{1},
   % keywordstyle=[2]\normalfont\texttt,
   morecomment=[l]{Input:},
   morecomment=[l]{Parameter},
   morecomment=[l]{Parameters:},
   morecomment=[l]{Returns:},
   morecomment=[l]{Effect:},
   morecomment=[l]{Pre-conditions:},
   morecomment=[l]{State:},
   morecomment=[l]{Messages:},
   morecomment=[l]{//},
   morecomment=[s]{(*}{*)},
   moredelim=[is]{(**}{**)},
   morestring=[b]",
   % morestring=[b]',
  }
  [keywords]

% \def\keyw#1{\lstinline!#1!}
% \def\jkeyw#1{\lstinline[style=javacode]!#1!}
% \def\pskeyw#1{\lstinline[style=pseudocode]!#1!}

%   \lstset{style=pseudocode}
%
%   \begin{document}
%
%   \begin{lstlisting}
%   r = 0
%   i = n*n
%   while i > 1 do
%       for j = 1 to i do
%           r = r + 1
%       i = i/2
%   \end{lstlisting}
%--------------------
%End Example

\usepackage{listings}
\lstdefinestyle{ccode}{
	language=C,
	keywordstyle=\ttfamily\underbar,
	morekeywords={DEFINE},
	keywordstyle=[2]\texttt,
	morekeywords=[2]{GLfloat},
	identifierstyle=\texttt,
	columns=flexible,
}
\lstdefinestyle{pythoncode}{
		language=Python,
        keywordstyle=\ttfamily\underbar,
        keywordstyle=[2]\texttt,
        identifierstyle=\texttt,
		commentstyle=\texttt,
		stringstyle=\texttt,
		tabsize=4,
		showstringspaces=false,
}
\lstdefinestyle{javacode}{
		language=Java,
        keywordstyle=\ttfamily\underbar,
		morekeywords={enum,assert},
        keywordstyle=[2]\texttt,
        identifierstyle=\texttt,
		commentstyle=\texttt,
		stringstyle=\texttt,
		tabsize=4,
		showstringspaces=false,
}
\lstdefinestyle{pseudocode}{
		language=pseudo,
        keywordstyle=\normalfont\textbf,
        keywordstyle=[2]\normalfont\textsf,
		keywordstyle=[3]\normalfont\textsf,
        identifierstyle=\normalfont\textit,
		commentstyle=\textrm,
		stringstyle=\texttt,
		tabsize=4,
		showstringspaces=false,
}
\lstdefinelanguage{pseudo}%
  {%identifierstyle=\normalfont\textit,
   columns=fullflexible,
   mathescape=true,
%   numbers=left,
%   numberblanklines=false,
   breaklines=true,
   breakatwhitespace=true,
   morekeywords={if,then,else,elseif,endif,switch,endswitch,case,while,do,endw,repeat,until,for,foreach,in,endf,break,begin,end,function,return,or,and,to,by,input,output,continue,type,of},
   morekeywords=[2]{int, float, double, bool, char, string, seq, list, array, set, map, void, stack, collection},
   morekeywords=[3]{true, false, null},
   % keywordstyle=\normalfont\textbf,
   literate={<-}{{${}\leftarrow{}$}}{2} {[]}{[ ]}{2} {!=}{{${}\not={}$}}{1} {<=}{${}\leq{}$}{1} {>=}{${}\geq{}$}{1},
   % keywordstyle=[2]\normalfont\texttt,
   morecomment=[l]{Input:},
   morecomment=[l]{Parameter},
   morecomment=[l]{Parameters:},
   morecomment=[l]{Returns:},
   morecomment=[l]{Effect:},
   morecomment=[l]{Pre-conditions:},
   morecomment=[l]{State:},
   morecomment=[l]{Messages:},
   morecomment=[l]{//},
   morecomment=[s]{(*}{*)},
   moredelim=[is]{(**}{**)},
   morestring=[b]",
   % morestring=[b]',
  }
  [keywords]

% \def\keyw#1{\lstinline!#1!}
% \def\jkeyw#1{\lstinline[style=javacode]!#1!}
% \def\pskeyw#1{\lstinline[style=pseudocode]!#1!}


\setcounter{secnumdepth}{2}
\setcounter{tocdepth}{2}

\lstset{
    language=Java,
    showspaces=false,
    showstringspaces=false,
    showtabs=false,
    breaklines=true,
    numberstyle=\tiny,
    basicstyle=\tiny, stepnumber=1, frame=single,
    numbersep=4pt,
    tabsize=2,
    extendedchars=true,
    numbers=left
}




\usepackage{fancyhdr}
\pagestyle{fancy}
\lhead{}
\chead{}
\rhead{COMP 360: Matt Adelman, Evan Carmi, Adam Forbes}
\lfoot{}
\cfoot{\thepage}
\rfoot{}

\newcommand{\HRule}{\rule{\linewidth}{0.5mm}}

\newcommand{\ty}[1]{\texttt{#1}}

\begin{document}

\begin{titlepage}

\begin{center}
% Upper part of the page
\textsc{\LARGE COMP 360}\\[1.5cm]
\textsc{\Large Final Project}\\[0.5cm]
% Title
\HRule \\[0.4cm]
{ \huge \bfseries RSA Moduli Pitfalls}\\[0.4cm]
\HRule \\[1.5cm]
% Author and supervisor
\begin{minipage}{0.4\textwidth}
\begin{center} \large
\emph{Authors:}\\
Matt Adelman\\Evan Carmi\\Adam Forbes
\end{center}
\end{minipage}
\vfill
% Bottom of the page
{\large \today}
\end{center}
\end{titlepage}

\tableofcontents

\newpage

\begin{abstract}
\noindent The algorithm RSA is one of the most widely implemented algorithms in the
world. The security of RSA implementations has long been a question of study.
In this paper we aim to examine a vulnerability in public keys of RSA found be
Lenstra et al. which states that RSA offers only a 99.8\% security because of
poorly generated prime numbers. We collect data on the top 1 million websites
and attempt to reanalyze this claim.
\\
\\
\textbf{Keywords:} RSA, security, prime numbers,  Euclidean algorithm, factoring, data collection
\end{abstract}

\begin{multicols}{2}
\section{History:}
\lettrine[nindent=0em,lines=3]{R} SA is a public key encryption algorithm 
developed by Ron Rivest, Adi Shamir, 
and Leonard Adleman at MIT in 1977. Rivest and Shamir were both computer
scientists that were working on an ``unbreakable'' public key encryption method.
Rivest and Shamir worked on many different codes, and would pose them as a
challenge to Adleman. Forty two of these codes were presented, and Adleman broke
them all. Finally on attempt number 43, they created what is now known as the
RSA scheme. Incidentally, English Professor Clifford Cocks developed the exact
same encryption system in 1973, but it was classified as top-secret, so it was
not released until 1997. The RSA algorithm was released for public domain in
1997.

The algorithm operates by using two distinct, large prime numbers to generate
public and private keys. Anyone can use the public key to encrypt a message, but
only someone with the private key can decrypt it. Breaking the RSA requires
factorization of the prime numbers used to create the keys. Since all known
prime factorization algorithms are exponential in time, if the prime numbers
are large enough RSA is secure.

\section{Original Ideas:}
When we first started out with this project, our ideas were fairly
straightforward. We were going to implement the RSA algorithm in JavaScript, put
it on a website, and address some of the common implementation issues that cause
security flaws. A few that we were going to address were timing attacks, public
exponent flaws and bad prime generation. This is where things got interesting.
We were given the paper by Lenstra et al ``Ron Was Wrong, Whit Is Right.'' While
more on this paper will be addressed in the next section, the experiments
performed by Lenstra exposed how big of an issue bad prime generation was. Our
goal then shifted away from rigorous implementation, although we still did
create a sufficient implementation, to recreating the experiments from the paper
and attempting to find some bad moduli of our own.

\section{Lit Review:}
As stated earlier, ``Ron was Wrong, Whit is Right'' was not the initial focus of
our project. However, upon further reading and the study of Lenstra et al.
proved to be instrumental in the direction of our project. The paper was
published on February 12, 2012 and created a splash in the field of
cryptography. Its most influential discovery was that two out of every thousand
RSA moduli offer no security at all. Leading to the conclusion that RSA offers a
disconcertingly inadequate 99.8\% security. This was discovered by utilizing the
Euclidean Algorithm and running it against their huge dataset of certificates in
order to find moduli with shared primes.

Their findings were shocking to us. We were inspired to recreate their
experiments on the state of the web nine months later in the hopes of finding
one of two possible conclusions. Either we find broken moduli or we find none.
The former would imply a lack of real-world implementation despite the influence
of their paper. The latter would indicate an improvement of online security
since the data collection of Lenstra et al. However, in order to begin our
recreation we needed to first review their work which will be summarized below.

They gathered their data from a collection of public-key databases. Of these the
two most significant databases being the SSL Observatory (a project run by the
Electronic Frontier Foundation) and one  from Massachusetts Institute of
Technology. Their method of collection differed greatly from ours as they ``did
not engage in web crawling, extensive ssh-session monitoring, or other data
activities that may be perceived as intrusive or aggressive'', a philosophy we
did not share. Their research led them to spend a long time discussing the
differences between the public keys they collected and the security differences
between them. They did extensive clustering of moduli to describe them as a
graph. However, many of their findings are only tangentially related to the work
we have done and therefore will be left out of this section.

The most pertinent information we are concerned with lies in the size of their
final dataset and the number of compromised RSA moduli agnostic of the types of
certificates they came from. They collected 11666704 public keys of which
6386984 had distinct RSA moduli with the rest being split evenly between
ElGamal and DSA. Of these 6386984 moduli they found 12934 different ones which
could be factored by 14901 distinct primes and were therefore compromised.

\section{Algorithm Implementation:}
For our implementation, we chose JavaScript as our environment because of its
ease for web development. We really wanted to end up with something that we
could show off and display in a public setting. The code to the implementation
of our RSA algorithm can be found in Appendix A. For our implementation, we
relied heavily on the BigInt library found here: 
\url{http://leemon.com/crypto/BigInt.js}.
While there were other BigInt libraries, out there but this is the one that had
all of the functions we needed, and was implemented in an easy format for us to
deal with. The main functions we used from this library were the following:
\ty{str2bigInt} which takes a string and a base that represents the big integer 
and converts it for us. 
\ty{inverseMod} which takes a big integer and a modulus as a big integer, and 
returns the big integer that is the inverse if one exists, otherwise \ty{null}. 
\ty{mult} which has the obvious function of multiplying two big integers.
\ty{greater} which returns a boolean depending on whether or not the first 
argument is greater than the second.
\ty{modPow(a,b,c)} which takes three big integers and returns $a^b \mod c$.  
\ty{addInt} which takes a big integer and an integer and adds the integer to 
the big integer. This was used in the computation of $\phi(n)$. 

The functions we created are as follows:
\ty{generate\_key(p:String, q:String, e:String) = [[n:BigInt, e:BigInt],
[n:BigInt, d:BigInt]]} where $p$ and $q$ are our distinct primes and $e$ is the
public exponent all represented as strings. We then return $n$ which is the
modulus, with $n = pq$. Together with $e$ the public exponent and $d$ the
private exponent. These together make up our keys.
\ty{crypt(key:[BigInt,BigInt], message:BigInt) = message$'$:BigInt} where the 
$key$ is a modulus with an exponent, and the $message$ is something to be 
encrypted or decrypted depending on what $key$ with which we call \ty{crypt}.

The next step for our implementation was getting everything to play nice with
our web implementation (found here: \url{http://carmi.github.com/rsa/code/} ).
Our plan was to create some helper functions that would access the data from
our form, check for errors on input and even randomly generate pseudo-random
primes based on a bit length. The code for these helper functions can be found
in Appendix A. Some of the big ones are the \ty{gen} function which takes the
information given in the form, tests for errors such as bad public exponents,
or if the public exponent is not relatively prime to $\phi(n)$. We also have
functions to encrypt and decrypt messages given that we have public and private
keys entered into our form.

To ensure that future changes do not introduce bugs into our already working
codebase we use tests. For our JavaScript implementation of RSA we used QUnit
tests, a testing framework in JavaScript used by jQuery among others. This
allows our tests to be run easily in the browser along side our implementation.
If future changes break a function, then the associated test will fail alerting
us to the portion of code that is broken.  
This included minor tests for generation, encryption, decryption, and
even a few number theory functions which were in our original implementation of
the RSA algorithm that only worked on integers. This is actually when we
realized we needed to switch to arbitrary precision integers, because the
exponents get very large, very quickly.

\section{Data Collection:}
Any analysis of RSA certificates requires a set of data with which to test. For
our project we needed a large set of data so as to best mimic the data that
Lenstra used in their analysis of public keys. Lenstra stated in
their paper that they used the Electronic Frontier Foundation's SSL Observatory
- \url{https://www.eff.org/observatory}. We initially looked at the SSL
Observatory as a possibly source of data for our project. However,
unfortunately the SSL Observatory has not been updated in a few years.
Furthermore, it includes no complete guide on how to use its source, and is
simply a collection of python scripts that interact with a \ty{MySQL} database
in distinct ways. Thus, we were unable to successfully use the SSL Observatory
source code as a means to gather RSA public keys. However, there exists data
dumps from the SSL Observatory from 2010. We gathered this data, available
through \ty{bittorrent}, and used it as a starting point.

In order to see the effectiveness and change since the Lenstra paper was
published in early 2012 we needed to scrap for fresh data. (The paper briefly
mentions that the latest data they looked at was from February 2012.) The SSL
Observatory scans all \ty{IPv4} space for servers that respond to https 
requests. We
did not have the computational time/power to do this, so we decided on the Alexa
top 1 million domains as a source of hosts. Then for each one of these domains
we attempted to connect via HTTPS and collect the public key used for the
connection. The specific method of collecting these moduli was the  This was
done by \ty{curl}'ing the domain at port 443 with a \ty{HTTP} request. If we 
received a
response, we used \ty{openssl s\_client} to get the decoded certificate and
\ty{openssl x509} to decode the certificate into a easily passable format.
However, we noticed that less than 10\% of domains responded to our requests. We
added some further error handling for these situations. We followed redirects
and tried a few common subdomains to improve our rate of responses to over 12\%.
Out of the 1000000 domains we received 125723 X.509 certificates each containing
a moduli. This was the source of our data for our experiments.

In order to deal with our large directory of certificates we needed combine
them into a single csv that for the sake of efficiency discards all the
impertinent data.  We wrote a parsing program in Java that reads in every file
and parses it into a combined csv file. The biggest challenge with dealing with
the certificates was handling the exceptions that inevitably arise when dealing
with large numbers of only mostly standardized data. 

Once we had all of the certificates in a csv file, the next step was to parse
out the information we needed. Since the modulus was in a different column for
each certificate file, therefore it was in a different column for each line in
the csv file.  However, there was the unique character pattern \ty{``$)$Modulus$($''} on each line of the file. Therefore we could match for that using the
\ty{substring} method of the \ty{String} class in Java. Once we had that, we
stripped the modulus of all extraneous characters such as white space and
colons. Then since all of the other information we needed in the csv file was
before the modulus, we just used a \ty{split(``,'')} on the line to split
around commas and easily obtained the domain name, and a unique identifier for
each modulus.

Finally we needed code to create a random sample of moduli to more swiftly run
an analysis on. We also wanted to make sure that these moduli were unique. We
did this by looping through and adding the moduli we chose to a set to make sure
we never chose two that were the same. The parsing and random sample getting
code is seen in Appendix A.


\section{Experiments:}
The main idea of our experiments is as follows: if we have two moduli $m, n$ we
can find if these two moduli share a prime $p$ simply by taking the $\gcd$ of
the two moduli. This is very fast and is linear in the number of bits. To do
around 100000 $\gcd$ operations on the machines we were using took around 5
seconds each. Therefore our code, seen in Appendix A, does the following: It
takes a file that contains the list of moduli, along with other identifying 
information. It also takes a smaller file which only contains a small section of
the moduli. This was done so the processing could be broken up onto multiple
machines. Once the two files are read into memory, the program stores the moduli
in an array, and a unique identifier corresponding to the moduli in another
array. We then loop through the data comparing every modulus in the smaller file
with every modulus in the bigger file.

We used Java for our experiments due to speed, and the ease with which we can
allocate memory to the virtual machine. The default JVM heap space is not large
enough to read in the files we were working with (some of which were over 100MB)
so we allocated extra memory with the \ty{java -Xmx2g} option which allocates an
extra 2GB of data to the virtual machine. We probably did not need this much
more allocated, but it was used just as a precaution.

The algorithm was designed to overlook all of moduli that are the same, because
while this is interesting information, it gave us no additional data. If the
algorithm came across two moduli $n,m$ such that $n \neq m$ and $\gcd(n,m) \neq
1$ then we know we have a prime in common and have factored the modulus,
rendering it insecure. Once it finds one, it prints out the $\gcd$, the unique
identifier for each modulus and the result of calling \ty{isProbablePrime(80)}
which returns true if there is less than a $2^{-80}$ chance that the $\gcd$ is
composite. We then can use this data to try and factor the modulus.

\section{Analysis of Data:}
Using the results of Lenstra we analyzed the significance
of our findings. Here is a summary of both of our analyses:

\textbf{Lenstra et al.}
\begin{compactitem}
\item 11666704 public keys
\item 6386984 distinct RSA moduli
\item 12934 distinct broken moduli
\item 14901 distinct primes
\end{compactitem}

\break
\textbf{Us}
\begin{compactitem}
\item 105984 distinct RSA moduli
\item $\sim$60000 moduli analyzed
\item 0 broken moduli
\item 0 primes
\end{compactitem}

These results are reassuring. They show that out of the 105984 distinct RSA
moduli observed by our scraping we could not find any that were compromised.
Unfortunately, due to a lack of time and resources (we ran analysis on six
computers for approximately two weeks) we were only able to verify around 60000
of the moduli against the set of 105984. Our analysis is ongoing and therefore
60000 is constantly increasing. 

The set of 60000 were spread over six clusters arranged alphabetically on their
url. We will assume that these were chosen randomly as the original list of the
top 1 million domain names was outputted alphabetically and not by ranking. In
addition to this prefixes to domains and subdomains further splits up the set of
certificates collected. 

Our first attempt at calculating the statistical significance of our results was
to ask the following question: if there is a .002 chance of finding a broken
moduli in the set of Lenstra et al. how many moduli should we expect to find?

\begin{compactitem}
\item Let $n$ be the size of our sample set of distinct moduli
\item Let $p$ be the probability of finding a broken moduli
\item Let $m$ be the expected number of broken moduli
\item $np = m$
\item $105984 \cdot .002 = 212$
\end{compactitem}


We would expect to find 212, and we found 0. Is this significant? Yes it is,
because the chance of finding no broken moduli is $(.998)^{60 000} = 
6.7995\cdot10^{-53}$.

However, there are many problems with this analysis. The largest being that we
were only able to scrape a dataset with a size that's 1.66\% of that analyzed by
Lenstra et al. This wouldn't be a large problem if you could identify the broken
moduli by themselves, but by the very nature of our analysis you need to run the
Euclidean Algorithm on every single moduli against every other moduli, hence the
unsurpassable amount of time and computing power required to factor the moduli.
This means that the chances of finding a broken moduli scale very badly with a
smaller sample size. 

Lenstra et al. found a large range of compromised moduli clusters. However, for
this analysis let's assume the worst case: every modulus has a pair that shares
a prime. In the case of the research done by Lenstra et al. we have 14901
distinct primes and therefore a total of 29802 pairs. This is no the actual case
because their research shows clusters of moduli that share the same prime. If we
were to include the statistical probabilities of every group size the model
would become too cumbersome and therefore analyzing the worst case scenario
provides a good estimate.

Due to a lack of probability experience we have written a program that simulates
our statistical problem and determines the expected number of pairs. The program
runs as follows:
\begin{compactitem}
\item Populate an array of size 6386984 with 0s
\item Loop 14901 times. Each time randomly assigning two unassigned elements to
be a and negative a. These will act as the pairs of moduli that share a prime.
\item Create another array of size 105984 and populate it with a random 
selection from the first array.
\item Count the number of pairs in the new array
\end{compactitem}
This program was run 100000 times and found that, on average, there are 8.188
pairs in the second array. Meaning that we should expect to find at 8 pairs in
our dataset - a fact we did not. Now that we know we should expect 8 pairs in 
our dataset we need to calculate the probability of our findings. This code is
seen in Appendix A.

First we need to calculate the probability any given moduli should be broken.
$p = 8/105984= 7.73\cdot10^{-5}$ But if we only completed analyzing 60000
moduli, how many should we find to be broken? \#broken = $(7.73\cdot10^{-5})
\cdot60000 = 4.638$ Let's round this up to a probable 5 broken moduli. 

Therefore we can say with confidence that 8 out of every 100000 should be
broken in our smaller set and we should expect to find a little under 4 broken
moduli in our analysis. Notice that this chance of pairs is based on the
research of Lenstra et al. and yet gives us a substantially worse chance of
finding broken moduli. We have a $0.0000773$ chance of finding a compromised
moduli out of 105984 whereas their research found a 0.002 chance of finding a
compromised moduli out of 6386984 moduli.

\section{Problems:}

Computation Power - The largest problem we encountered was computational power.
Simply put, we did not have the computing power to analyze all the data we
gathered. There primary function we wanted to do was run a $\gcd$ across
every moduli we encountered. And we couldn't do this in time. There were a few
methods we took to try and solve this. We split our data into multiple sections
and ran it across a few different machines. Nonetheless, we were unable to
process enough data fast enough. We simply needed a compute cluster, or more
time. At some points we had data running on five different machines.

SSL Observatory - We attempted to contact the EFF SSL Observatory maintainer
for help getting the Observatory running, and to ask if there was a newer data
dump than the 2010 data that was on bittorrent. We never received a response.
It's likely that the not-for-profit EFF didn't have anyone working on the
project who wanted to help us.

Same Moduli - Something we did not expect but ran into quite frequently were the
existence of many moduli that were the same. For instance out of the 125723
moduli that we collected, just over 104000 were unique. Therefore around 18\% of
the moduli we collected were not unique. In addition, around 10\% of that was
all from the same domain $*$.googleusercontent.com. The implication of this is
that we could not run as many comparisons as we would have liked, because we
had to throw out a lot of same data. The fact that Google uses the same moduli
for so many of its domains really speaks to the security of this algorithm

\section{Conclusion:}
We did not find any moduli that did not offer security. What are the chances of
this happening?  $p = (1-0.0000773)^{60000} = 0.00968$
There is a 1\% chance of not finding any moduli in our set, and therefore we can
say with 99\% confidence that our findings show an improvement in RSA security
because we were unable to find any broken moduli.

One possible explanation for this can be found in the bias of our domain names.
We scraped from the top 1 million most trafficked websites online, and therefore
one can hypothesize that there is a lot of overlap between this list and the
list of most maintained websites online. The highest probability of finding a
bad modulus may lie in looking at the top million least visited websites. 


\section{Further Research:}
The first priority would be to collect a larger dataset of moduli closer to that
of Lenstra et al. as it would greatly improve the statistical significance of
our research. With the data we collected, including the older 2010 data dump 
from the SSL Observatory, we have a large set of moduli. We discussed the 
possibility of creating a service that would check any new moduli against the 
entire every seen set of moduli to ensure there are never duplicates. This 
would, in effect, allow us verify that certificates are safe prior to them being
used by websites. In fact, we hope that some certificate authorities are already
doing this to ensure that within their own network of certificates there are no
duplicates. The fact that $\gcd$ is so fast makes a database, and a test like
this incredibly feasible.

\end{multicols}

\appendix
\section{Appendix of code:}

\lstinputlisting[caption=A ruby script that interfaces with unix command line
utilities to collect X.509 certificates given a list of hosts.,language=ruby]
{ssl_cert_scraper.rb}

\lstinputlisting[caption=A Java class that takes the csv file of certificates
and extracts all the useful information.,language=Java]
{code/data/FormatNewData.java}

\lstinputlisting[caption=A Java class to get the random sample of moduli.
,language=Java] {code/data/GetRandomSample.java}

\lstinputlisting[caption=A Java class that reads in two files containing our
moduli and other information and prints out a prime if we find one between two
moduli.,language=Java]{code/data/RandomSampleCheck.java}


\pagebreak

\lstinputlisting[caption=A Java class that simulates the probable number of
pairs of bad moduli we should have found.]{code/statTest.java}


\end{document}
