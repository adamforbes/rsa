\documentclass[12pt]{article}

\usepackage{amsmath}
\usepackage{amssymb}
\usepackage{amsfonts}

\setcounter{secnumdepth}{2}
\setcounter{tocdepth}{2}


\usepackage[margin=1in]{geometry}

\usepackage{fancyhdr}
\pagestyle{fancy}
\lhead{}
\chead{}
\rhead{COMP 360: Matt Adelman, Evan Carmi, Adam Forbes}
\lfoot{}
\cfoot{\thepage}
\rfoot{}

\newcommand{\HRule}{\rule{\linewidth}{0.5mm}}

\begin{document}

\begin{titlepage}

\begin{center}
% Upper part of the page
\textsc{\LARGE COMP 360}\\[1.5cm]
\textsc{\Large Final Project}\\[0.5cm]
% Title
\HRule \\[0.4cm]
{ \huge \bfseries RSA Moduli Pitfalls}\\[0.4cm]
\HRule \\[1.5cm]
% Author and supervisor
\begin{minipage}{0.4\textwidth}
\begin{center} \large
\emph{Authors:}\\
Matt Adelman\\Evan Carmi\\Adam Forbes
\end{center}
\end{minipage}
\vfill
% Bottom of the page
{\large \today}
\end{center}
\end{titlepage}

\tableofcontents

\newpage

\begin{@twocolumnfalse}

\begin{abstract}
  Hello world, this is the abstract.  Hello world, this is the abstract.
  Hello world, this is the abstract.  Hello world, this is the abstract.  Hello world, this is the abstract.  Hello world, this is the abstract.  Hello world, this is the abstract.  Hello world, this is the abstract.  Hello world, this is the abstract.
\end{abstract}
\end{@twocolumnfalse}


\section{History:}
\indent\indent
RSA is a public key encryption algorithm developed by Ron Rivest, Adi Shamir, 
and Leonard Adleman at MIT in 1977. Rivest and Shamir are both computer
scientists that were working on an ``unbreakable'' public key encryption method.
Rivest and Shamir worked on many different codes, and would pose them as a
challenge to Adleman. Forty two of these codes were presented, and Adleman broke
them all. Finally on attempty number 43, they created what is now known as the
RSA scheme. Incidentally, English Professor CLifford Cocks developed the exact
same encryption system in 1973, but it was classified as top-secret, so it was
not released until 1997. The RSA algorithm was released for public domain in
1997.\\\indent
The algorithm operates by using two distinct, large prime numbers to generate
public and private keys. Anyone can use the public key to encrypt a message, but
only someone with the private key can decrypt it. The algorithm is hard to
break, because if the prime numbers are large enough, the factorization is
exponential in time.

\section{Original Ideas:}

\section{Algorithm Implementation:}
For 

\section{Data Collection:}
Any analysis of RSA certificates requires a set of data with which to test. For our project we needed a large set of data so as to best mimic the data that Lenstra
et al used in their analysis of public keys. Lenstra et al stated in their paper that they used the Electronic Frontier Foundation's SSL Observatory - https://www.eff.org/observatory. We initially looked at the SSL Observatory as a possibly source of data for our project. However, unfortunately the SSL Observatory has not been updated in a few years. Furthermore, it includes no complete guide on how to use its source, and is simply a collection of python scripts that interact with a MySQL database in distinct ways. Thus, we were unable to successfully use the SSL Observatory source code as a means to gather RSA public keys. However, there exists data dumps from the SSL Observatory from 2010. We gathered this data, available through bittorrent, and used it as a starting point.

In order to see the effectiveness and change since the Lenstra paper was published in early 2012 we needed to scrap for fresh data. (The paper briefly mentions that the latest data they looked at was from February 2012.) The SSL Observatory scans all IPv4 space for servers that respond to https requests. We did not have the computational time/power to do this, so we decided on the Alexa top 1 million domains as a source of hosts. Then for each one of these domains we attempted to connect via HTTPS and collect the public key used for the connection. The specific method of collecting these moduli was the  This was done by curl'ing the domain at port 443 with a HTTP request. If we received a response, we used \ty{openssl s_client} to get the decoded certificate and \ty{$openssl x509} to decode the certificate into a easily passable format.  However, we noticed that less than 10\% of domains responded to our requests. We added some further error handling for these situations. We followed redirects and tried a few common subdomains to improve our rate of responses to over 12\%. Out of the 1000000 domains we received 125723 X.509 certificates each containing a moduli. This was the source of our data for our experiments.

\section{Analysis:}

\section{Problems:}

\section{Conclusion:}

\section{Further Research:}

\appendix
\section{Appendix of code:}


\end{document}

Abstract (All):

History (Matt):

Lit Review (Adam):
 - RWWWIR

Original ideas:

Algorithm Implementaion (Matt):
 - What we did
 - Big int package used
 - Why web implementaion

Data collection (Evan):
- SSL observatoru 2010
- new data collection
- parsig new data
- analysing new data (and old data)

analysis

conclusion, things to do better

further research/work
   - database of moduli that you can check on

- code



Possible tables and graphs
- top ten shared moduli
- number of new data collected
    - distribution of key size
-Time complexity of expected vs. actual
