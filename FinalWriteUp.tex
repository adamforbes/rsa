\documentclass[12pt]{article}

\usepackage[margin=1in]{geometry}

\usepackage{amsmath}
\usepackage{amssymb}
\usepackage{amsfonts}
\usepackage{listings}

% listings.tex - A LaTeX file that configures some settings for the listings package.
% Author: Evan Carmi http://ecarmi.org/
% Based off configurations from Professor Norman Danner (Wesleyan University).
%
% Usage: to include this file in another .tex file add:
%   % listings.tex - A LaTeX file that configures some settings for the listings package.
% Author: Evan Carmi http://ecarmi.org/
% Based off configurations from Professor Norman Danner (Wesleyan University).
%
% Usage: to include this file in another .tex file add:
%   % listings.tex - A LaTeX file that configures some settings for the listings package.
% Author: Evan Carmi http://ecarmi.org/
% Based off configurations from Professor Norman Danner (Wesleyan University).
%
% Usage: to include this file in another .tex file add:
%   \input{/path/to/listings.tex}
%   and the style for the code like:
%   \lstset{style=pseudocode}
% to the preamble of that document.
%
% Example:
%--------------------
%   \input{/home/evan/tex/listings.tex}
%   \lstset{style=pseudocode}
%
%   \begin{document}
%
%   \begin{lstlisting}
%   r = 0
%   i = n*n
%   while i > 1 do
%       for j = 1 to i do
%           r = r + 1
%       i = i/2
%   \end{lstlisting}
%--------------------
%End Example

\usepackage{listings}
\lstdefinestyle{ccode}{
	language=C,
	keywordstyle=\ttfamily\underbar,
	morekeywords={DEFINE},
	keywordstyle=[2]\texttt,
	morekeywords=[2]{GLfloat},
	identifierstyle=\texttt,
	columns=flexible,
}
\lstdefinestyle{pythoncode}{
		language=Python,
        keywordstyle=\ttfamily\underbar,
        keywordstyle=[2]\texttt,
        identifierstyle=\texttt,
		commentstyle=\texttt,
		stringstyle=\texttt,
		tabsize=4,
		showstringspaces=false,
}
\lstdefinestyle{javacode}{
		language=Java,
        keywordstyle=\ttfamily\underbar,
		morekeywords={enum,assert},
        keywordstyle=[2]\texttt,
        identifierstyle=\texttt,
		commentstyle=\texttt,
		stringstyle=\texttt,
		tabsize=4,
		showstringspaces=false,
}
\lstdefinestyle{pseudocode}{
		language=pseudo,
        keywordstyle=\normalfont\textbf,
        keywordstyle=[2]\normalfont\textsf,
		keywordstyle=[3]\normalfont\textsf,
        identifierstyle=\normalfont\textit,
		commentstyle=\textrm,
		stringstyle=\texttt,
		tabsize=4,
		showstringspaces=false,
}
\lstdefinelanguage{pseudo}%
  {%identifierstyle=\normalfont\textit,
   columns=fullflexible,
   mathescape=true,
%   numbers=left,
%   numberblanklines=false,
   breaklines=true,
   breakatwhitespace=true,
   morekeywords={if,then,else,elseif,endif,switch,endswitch,case,while,do,endw,repeat,until,for,foreach,in,endf,break,begin,end,function,return,or,and,to,by,input,output,continue,type,of},
   morekeywords=[2]{int, float, double, bool, char, string, seq, list, array, set, map, void, stack, collection},
   morekeywords=[3]{true, false, null},
   % keywordstyle=\normalfont\textbf,
   literate={<-}{{${}\leftarrow{}$}}{2} {[]}{[ ]}{2} {!=}{{${}\not={}$}}{1} {<=}{${}\leq{}$}{1} {>=}{${}\geq{}$}{1},
   % keywordstyle=[2]\normalfont\texttt,
   morecomment=[l]{Input:},
   morecomment=[l]{Parameter},
   morecomment=[l]{Parameters:},
   morecomment=[l]{Returns:},
   morecomment=[l]{Effect:},
   morecomment=[l]{Pre-conditions:},
   morecomment=[l]{State:},
   morecomment=[l]{Messages:},
   morecomment=[l]{//},
   morecomment=[s]{(*}{*)},
   moredelim=[is]{(**}{**)},
   morestring=[b]",
   % morestring=[b]',
  }
  [keywords]

% \def\keyw#1{\lstinline!#1!}
% \def\jkeyw#1{\lstinline[style=javacode]!#1!}
% \def\pskeyw#1{\lstinline[style=pseudocode]!#1!}

%   and the style for the code like:
%   \lstset{style=pseudocode}
% to the preamble of that document.
%
% Example:
%--------------------
%   % listings.tex - A LaTeX file that configures some settings for the listings package.
% Author: Evan Carmi http://ecarmi.org/
% Based off configurations from Professor Norman Danner (Wesleyan University).
%
% Usage: to include this file in another .tex file add:
%   \input{/path/to/listings.tex}
%   and the style for the code like:
%   \lstset{style=pseudocode}
% to the preamble of that document.
%
% Example:
%--------------------
%   \input{/home/evan/tex/listings.tex}
%   \lstset{style=pseudocode}
%
%   \begin{document}
%
%   \begin{lstlisting}
%   r = 0
%   i = n*n
%   while i > 1 do
%       for j = 1 to i do
%           r = r + 1
%       i = i/2
%   \end{lstlisting}
%--------------------
%End Example

\usepackage{listings}
\lstdefinestyle{ccode}{
	language=C,
	keywordstyle=\ttfamily\underbar,
	morekeywords={DEFINE},
	keywordstyle=[2]\texttt,
	morekeywords=[2]{GLfloat},
	identifierstyle=\texttt,
	columns=flexible,
}
\lstdefinestyle{pythoncode}{
		language=Python,
        keywordstyle=\ttfamily\underbar,
        keywordstyle=[2]\texttt,
        identifierstyle=\texttt,
		commentstyle=\texttt,
		stringstyle=\texttt,
		tabsize=4,
		showstringspaces=false,
}
\lstdefinestyle{javacode}{
		language=Java,
        keywordstyle=\ttfamily\underbar,
		morekeywords={enum,assert},
        keywordstyle=[2]\texttt,
        identifierstyle=\texttt,
		commentstyle=\texttt,
		stringstyle=\texttt,
		tabsize=4,
		showstringspaces=false,
}
\lstdefinestyle{pseudocode}{
		language=pseudo,
        keywordstyle=\normalfont\textbf,
        keywordstyle=[2]\normalfont\textsf,
		keywordstyle=[3]\normalfont\textsf,
        identifierstyle=\normalfont\textit,
		commentstyle=\textrm,
		stringstyle=\texttt,
		tabsize=4,
		showstringspaces=false,
}
\lstdefinelanguage{pseudo}%
  {%identifierstyle=\normalfont\textit,
   columns=fullflexible,
   mathescape=true,
%   numbers=left,
%   numberblanklines=false,
   breaklines=true,
   breakatwhitespace=true,
   morekeywords={if,then,else,elseif,endif,switch,endswitch,case,while,do,endw,repeat,until,for,foreach,in,endf,break,begin,end,function,return,or,and,to,by,input,output,continue,type,of},
   morekeywords=[2]{int, float, double, bool, char, string, seq, list, array, set, map, void, stack, collection},
   morekeywords=[3]{true, false, null},
   % keywordstyle=\normalfont\textbf,
   literate={<-}{{${}\leftarrow{}$}}{2} {[]}{[ ]}{2} {!=}{{${}\not={}$}}{1} {<=}{${}\leq{}$}{1} {>=}{${}\geq{}$}{1},
   % keywordstyle=[2]\normalfont\texttt,
   morecomment=[l]{Input:},
   morecomment=[l]{Parameter},
   morecomment=[l]{Parameters:},
   morecomment=[l]{Returns:},
   morecomment=[l]{Effect:},
   morecomment=[l]{Pre-conditions:},
   morecomment=[l]{State:},
   morecomment=[l]{Messages:},
   morecomment=[l]{//},
   morecomment=[s]{(*}{*)},
   moredelim=[is]{(**}{**)},
   morestring=[b]",
   % morestring=[b]',
  }
  [keywords]

% \def\keyw#1{\lstinline!#1!}
% \def\jkeyw#1{\lstinline[style=javacode]!#1!}
% \def\pskeyw#1{\lstinline[style=pseudocode]!#1!}

%   \lstset{style=pseudocode}
%
%   \begin{document}
%
%   \begin{lstlisting}
%   r = 0
%   i = n*n
%   while i > 1 do
%       for j = 1 to i do
%           r = r + 1
%       i = i/2
%   \end{lstlisting}
%--------------------
%End Example

\usepackage{listings}
\lstdefinestyle{ccode}{
	language=C,
	keywordstyle=\ttfamily\underbar,
	morekeywords={DEFINE},
	keywordstyle=[2]\texttt,
	morekeywords=[2]{GLfloat},
	identifierstyle=\texttt,
	columns=flexible,
}
\lstdefinestyle{pythoncode}{
		language=Python,
        keywordstyle=\ttfamily\underbar,
        keywordstyle=[2]\texttt,
        identifierstyle=\texttt,
		commentstyle=\texttt,
		stringstyle=\texttt,
		tabsize=4,
		showstringspaces=false,
}
\lstdefinestyle{javacode}{
		language=Java,
        keywordstyle=\ttfamily\underbar,
		morekeywords={enum,assert},
        keywordstyle=[2]\texttt,
        identifierstyle=\texttt,
		commentstyle=\texttt,
		stringstyle=\texttt,
		tabsize=4,
		showstringspaces=false,
}
\lstdefinestyle{pseudocode}{
		language=pseudo,
        keywordstyle=\normalfont\textbf,
        keywordstyle=[2]\normalfont\textsf,
		keywordstyle=[3]\normalfont\textsf,
        identifierstyle=\normalfont\textit,
		commentstyle=\textrm,
		stringstyle=\texttt,
		tabsize=4,
		showstringspaces=false,
}
\lstdefinelanguage{pseudo}%
  {%identifierstyle=\normalfont\textit,
   columns=fullflexible,
   mathescape=true,
%   numbers=left,
%   numberblanklines=false,
   breaklines=true,
   breakatwhitespace=true,
   morekeywords={if,then,else,elseif,endif,switch,endswitch,case,while,do,endw,repeat,until,for,foreach,in,endf,break,begin,end,function,return,or,and,to,by,input,output,continue,type,of},
   morekeywords=[2]{int, float, double, bool, char, string, seq, list, array, set, map, void, stack, collection},
   morekeywords=[3]{true, false, null},
   % keywordstyle=\normalfont\textbf,
   literate={<-}{{${}\leftarrow{}$}}{2} {[]}{[ ]}{2} {!=}{{${}\not={}$}}{1} {<=}{${}\leq{}$}{1} {>=}{${}\geq{}$}{1},
   % keywordstyle=[2]\normalfont\texttt,
   morecomment=[l]{Input:},
   morecomment=[l]{Parameter},
   morecomment=[l]{Parameters:},
   morecomment=[l]{Returns:},
   morecomment=[l]{Effect:},
   morecomment=[l]{Pre-conditions:},
   morecomment=[l]{State:},
   morecomment=[l]{Messages:},
   morecomment=[l]{//},
   morecomment=[s]{(*}{*)},
   moredelim=[is]{(**}{**)},
   morestring=[b]",
   % morestring=[b]',
  }
  [keywords]

% \def\keyw#1{\lstinline!#1!}
% \def\jkeyw#1{\lstinline[style=javacode]!#1!}
% \def\pskeyw#1{\lstinline[style=pseudocode]!#1!}

%   and the style for the code like:
%   \lstset{style=pseudocode}
% to the preamble of that document.
%
% Example:
%--------------------
%   % listings.tex - A LaTeX file that configures some settings for the listings package.
% Author: Evan Carmi http://ecarmi.org/
% Based off configurations from Professor Norman Danner (Wesleyan University).
%
% Usage: to include this file in another .tex file add:
%   % listings.tex - A LaTeX file that configures some settings for the listings package.
% Author: Evan Carmi http://ecarmi.org/
% Based off configurations from Professor Norman Danner (Wesleyan University).
%
% Usage: to include this file in another .tex file add:
%   \input{/path/to/listings.tex}
%   and the style for the code like:
%   \lstset{style=pseudocode}
% to the preamble of that document.
%
% Example:
%--------------------
%   \input{/home/evan/tex/listings.tex}
%   \lstset{style=pseudocode}
%
%   \begin{document}
%
%   \begin{lstlisting}
%   r = 0
%   i = n*n
%   while i > 1 do
%       for j = 1 to i do
%           r = r + 1
%       i = i/2
%   \end{lstlisting}
%--------------------
%End Example

\usepackage{listings}
\lstdefinestyle{ccode}{
	language=C,
	keywordstyle=\ttfamily\underbar,
	morekeywords={DEFINE},
	keywordstyle=[2]\texttt,
	morekeywords=[2]{GLfloat},
	identifierstyle=\texttt,
	columns=flexible,
}
\lstdefinestyle{pythoncode}{
		language=Python,
        keywordstyle=\ttfamily\underbar,
        keywordstyle=[2]\texttt,
        identifierstyle=\texttt,
		commentstyle=\texttt,
		stringstyle=\texttt,
		tabsize=4,
		showstringspaces=false,
}
\lstdefinestyle{javacode}{
		language=Java,
        keywordstyle=\ttfamily\underbar,
		morekeywords={enum,assert},
        keywordstyle=[2]\texttt,
        identifierstyle=\texttt,
		commentstyle=\texttt,
		stringstyle=\texttt,
		tabsize=4,
		showstringspaces=false,
}
\lstdefinestyle{pseudocode}{
		language=pseudo,
        keywordstyle=\normalfont\textbf,
        keywordstyle=[2]\normalfont\textsf,
		keywordstyle=[3]\normalfont\textsf,
        identifierstyle=\normalfont\textit,
		commentstyle=\textrm,
		stringstyle=\texttt,
		tabsize=4,
		showstringspaces=false,
}
\lstdefinelanguage{pseudo}%
  {%identifierstyle=\normalfont\textit,
   columns=fullflexible,
   mathescape=true,
%   numbers=left,
%   numberblanklines=false,
   breaklines=true,
   breakatwhitespace=true,
   morekeywords={if,then,else,elseif,endif,switch,endswitch,case,while,do,endw,repeat,until,for,foreach,in,endf,break,begin,end,function,return,or,and,to,by,input,output,continue,type,of},
   morekeywords=[2]{int, float, double, bool, char, string, seq, list, array, set, map, void, stack, collection},
   morekeywords=[3]{true, false, null},
   % keywordstyle=\normalfont\textbf,
   literate={<-}{{${}\leftarrow{}$}}{2} {[]}{[ ]}{2} {!=}{{${}\not={}$}}{1} {<=}{${}\leq{}$}{1} {>=}{${}\geq{}$}{1},
   % keywordstyle=[2]\normalfont\texttt,
   morecomment=[l]{Input:},
   morecomment=[l]{Parameter},
   morecomment=[l]{Parameters:},
   morecomment=[l]{Returns:},
   morecomment=[l]{Effect:},
   morecomment=[l]{Pre-conditions:},
   morecomment=[l]{State:},
   morecomment=[l]{Messages:},
   morecomment=[l]{//},
   morecomment=[s]{(*}{*)},
   moredelim=[is]{(**}{**)},
   morestring=[b]",
   % morestring=[b]',
  }
  [keywords]

% \def\keyw#1{\lstinline!#1!}
% \def\jkeyw#1{\lstinline[style=javacode]!#1!}
% \def\pskeyw#1{\lstinline[style=pseudocode]!#1!}

%   and the style for the code like:
%   \lstset{style=pseudocode}
% to the preamble of that document.
%
% Example:
%--------------------
%   % listings.tex - A LaTeX file that configures some settings for the listings package.
% Author: Evan Carmi http://ecarmi.org/
% Based off configurations from Professor Norman Danner (Wesleyan University).
%
% Usage: to include this file in another .tex file add:
%   \input{/path/to/listings.tex}
%   and the style for the code like:
%   \lstset{style=pseudocode}
% to the preamble of that document.
%
% Example:
%--------------------
%   \input{/home/evan/tex/listings.tex}
%   \lstset{style=pseudocode}
%
%   \begin{document}
%
%   \begin{lstlisting}
%   r = 0
%   i = n*n
%   while i > 1 do
%       for j = 1 to i do
%           r = r + 1
%       i = i/2
%   \end{lstlisting}
%--------------------
%End Example

\usepackage{listings}
\lstdefinestyle{ccode}{
	language=C,
	keywordstyle=\ttfamily\underbar,
	morekeywords={DEFINE},
	keywordstyle=[2]\texttt,
	morekeywords=[2]{GLfloat},
	identifierstyle=\texttt,
	columns=flexible,
}
\lstdefinestyle{pythoncode}{
		language=Python,
        keywordstyle=\ttfamily\underbar,
        keywordstyle=[2]\texttt,
        identifierstyle=\texttt,
		commentstyle=\texttt,
		stringstyle=\texttt,
		tabsize=4,
		showstringspaces=false,
}
\lstdefinestyle{javacode}{
		language=Java,
        keywordstyle=\ttfamily\underbar,
		morekeywords={enum,assert},
        keywordstyle=[2]\texttt,
        identifierstyle=\texttt,
		commentstyle=\texttt,
		stringstyle=\texttt,
		tabsize=4,
		showstringspaces=false,
}
\lstdefinestyle{pseudocode}{
		language=pseudo,
        keywordstyle=\normalfont\textbf,
        keywordstyle=[2]\normalfont\textsf,
		keywordstyle=[3]\normalfont\textsf,
        identifierstyle=\normalfont\textit,
		commentstyle=\textrm,
		stringstyle=\texttt,
		tabsize=4,
		showstringspaces=false,
}
\lstdefinelanguage{pseudo}%
  {%identifierstyle=\normalfont\textit,
   columns=fullflexible,
   mathescape=true,
%   numbers=left,
%   numberblanklines=false,
   breaklines=true,
   breakatwhitespace=true,
   morekeywords={if,then,else,elseif,endif,switch,endswitch,case,while,do,endw,repeat,until,for,foreach,in,endf,break,begin,end,function,return,or,and,to,by,input,output,continue,type,of},
   morekeywords=[2]{int, float, double, bool, char, string, seq, list, array, set, map, void, stack, collection},
   morekeywords=[3]{true, false, null},
   % keywordstyle=\normalfont\textbf,
   literate={<-}{{${}\leftarrow{}$}}{2} {[]}{[ ]}{2} {!=}{{${}\not={}$}}{1} {<=}{${}\leq{}$}{1} {>=}{${}\geq{}$}{1},
   % keywordstyle=[2]\normalfont\texttt,
   morecomment=[l]{Input:},
   morecomment=[l]{Parameter},
   morecomment=[l]{Parameters:},
   morecomment=[l]{Returns:},
   morecomment=[l]{Effect:},
   morecomment=[l]{Pre-conditions:},
   morecomment=[l]{State:},
   morecomment=[l]{Messages:},
   morecomment=[l]{//},
   morecomment=[s]{(*}{*)},
   moredelim=[is]{(**}{**)},
   morestring=[b]",
   % morestring=[b]',
  }
  [keywords]

% \def\keyw#1{\lstinline!#1!}
% \def\jkeyw#1{\lstinline[style=javacode]!#1!}
% \def\pskeyw#1{\lstinline[style=pseudocode]!#1!}

%   \lstset{style=pseudocode}
%
%   \begin{document}
%
%   \begin{lstlisting}
%   r = 0
%   i = n*n
%   while i > 1 do
%       for j = 1 to i do
%           r = r + 1
%       i = i/2
%   \end{lstlisting}
%--------------------
%End Example

\usepackage{listings}
\lstdefinestyle{ccode}{
	language=C,
	keywordstyle=\ttfamily\underbar,
	morekeywords={DEFINE},
	keywordstyle=[2]\texttt,
	morekeywords=[2]{GLfloat},
	identifierstyle=\texttt,
	columns=flexible,
}
\lstdefinestyle{pythoncode}{
		language=Python,
        keywordstyle=\ttfamily\underbar,
        keywordstyle=[2]\texttt,
        identifierstyle=\texttt,
		commentstyle=\texttt,
		stringstyle=\texttt,
		tabsize=4,
		showstringspaces=false,
}
\lstdefinestyle{javacode}{
		language=Java,
        keywordstyle=\ttfamily\underbar,
		morekeywords={enum,assert},
        keywordstyle=[2]\texttt,
        identifierstyle=\texttt,
		commentstyle=\texttt,
		stringstyle=\texttt,
		tabsize=4,
		showstringspaces=false,
}
\lstdefinestyle{pseudocode}{
		language=pseudo,
        keywordstyle=\normalfont\textbf,
        keywordstyle=[2]\normalfont\textsf,
		keywordstyle=[3]\normalfont\textsf,
        identifierstyle=\normalfont\textit,
		commentstyle=\textrm,
		stringstyle=\texttt,
		tabsize=4,
		showstringspaces=false,
}
\lstdefinelanguage{pseudo}%
  {%identifierstyle=\normalfont\textit,
   columns=fullflexible,
   mathescape=true,
%   numbers=left,
%   numberblanklines=false,
   breaklines=true,
   breakatwhitespace=true,
   morekeywords={if,then,else,elseif,endif,switch,endswitch,case,while,do,endw,repeat,until,for,foreach,in,endf,break,begin,end,function,return,or,and,to,by,input,output,continue,type,of},
   morekeywords=[2]{int, float, double, bool, char, string, seq, list, array, set, map, void, stack, collection},
   morekeywords=[3]{true, false, null},
   % keywordstyle=\normalfont\textbf,
   literate={<-}{{${}\leftarrow{}$}}{2} {[]}{[ ]}{2} {!=}{{${}\not={}$}}{1} {<=}{${}\leq{}$}{1} {>=}{${}\geq{}$}{1},
   % keywordstyle=[2]\normalfont\texttt,
   morecomment=[l]{Input:},
   morecomment=[l]{Parameter},
   morecomment=[l]{Parameters:},
   morecomment=[l]{Returns:},
   morecomment=[l]{Effect:},
   morecomment=[l]{Pre-conditions:},
   morecomment=[l]{State:},
   morecomment=[l]{Messages:},
   morecomment=[l]{//},
   morecomment=[s]{(*}{*)},
   moredelim=[is]{(**}{**)},
   morestring=[b]",
   % morestring=[b]',
  }
  [keywords]

% \def\keyw#1{\lstinline!#1!}
% \def\jkeyw#1{\lstinline[style=javacode]!#1!}
% \def\pskeyw#1{\lstinline[style=pseudocode]!#1!}

%   \lstset{style=pseudocode}
%
%   \begin{document}
%
%   \begin{lstlisting}
%   r = 0
%   i = n*n
%   while i > 1 do
%       for j = 1 to i do
%           r = r + 1
%       i = i/2
%   \end{lstlisting}
%--------------------
%End Example

\usepackage{listings}
\lstdefinestyle{ccode}{
	language=C,
	keywordstyle=\ttfamily\underbar,
	morekeywords={DEFINE},
	keywordstyle=[2]\texttt,
	morekeywords=[2]{GLfloat},
	identifierstyle=\texttt,
	columns=flexible,
}
\lstdefinestyle{pythoncode}{
		language=Python,
        keywordstyle=\ttfamily\underbar,
        keywordstyle=[2]\texttt,
        identifierstyle=\texttt,
		commentstyle=\texttt,
		stringstyle=\texttt,
		tabsize=4,
		showstringspaces=false,
}
\lstdefinestyle{javacode}{
		language=Java,
        keywordstyle=\ttfamily\underbar,
		morekeywords={enum,assert},
        keywordstyle=[2]\texttt,
        identifierstyle=\texttt,
		commentstyle=\texttt,
		stringstyle=\texttt,
		tabsize=4,
		showstringspaces=false,
}
\lstdefinestyle{pseudocode}{
		language=pseudo,
        keywordstyle=\normalfont\textbf,
        keywordstyle=[2]\normalfont\textsf,
		keywordstyle=[3]\normalfont\textsf,
        identifierstyle=\normalfont\textit,
		commentstyle=\textrm,
		stringstyle=\texttt,
		tabsize=4,
		showstringspaces=false,
}
\lstdefinelanguage{pseudo}%
  {%identifierstyle=\normalfont\textit,
   columns=fullflexible,
   mathescape=true,
%   numbers=left,
%   numberblanklines=false,
   breaklines=true,
   breakatwhitespace=true,
   morekeywords={if,then,else,elseif,endif,switch,endswitch,case,while,do,endw,repeat,until,for,foreach,in,endf,break,begin,end,function,return,or,and,to,by,input,output,continue,type,of},
   morekeywords=[2]{int, float, double, bool, char, string, seq, list, array, set, map, void, stack, collection},
   morekeywords=[3]{true, false, null},
   % keywordstyle=\normalfont\textbf,
   literate={<-}{{${}\leftarrow{}$}}{2} {[]}{[ ]}{2} {!=}{{${}\not={}$}}{1} {<=}{${}\leq{}$}{1} {>=}{${}\geq{}$}{1},
   % keywordstyle=[2]\normalfont\texttt,
   morecomment=[l]{Input:},
   morecomment=[l]{Parameter},
   morecomment=[l]{Parameters:},
   morecomment=[l]{Returns:},
   morecomment=[l]{Effect:},
   morecomment=[l]{Pre-conditions:},
   morecomment=[l]{State:},
   morecomment=[l]{Messages:},
   morecomment=[l]{//},
   morecomment=[s]{(*}{*)},
   moredelim=[is]{(**}{**)},
   morestring=[b]",
   % morestring=[b]',
  }
  [keywords]

% \def\keyw#1{\lstinline!#1!}
% \def\jkeyw#1{\lstinline[style=javacode]!#1!}
% \def\pskeyw#1{\lstinline[style=pseudocode]!#1!}




\setcounter{secnumdepth}{2}
\setcounter{tocdepth}{2}

\lstset{
    language=c,
    showspaces=false,
    showstringspaces=false,
    showtabs=false,
    breaklines=true,
    numberstyle=\tiny,
    basicstyle=\tiny\sffamily, stepnumber=1, frame=single,
    numbersep=4pt,
    tabsize=2,
    extendedchars=true,
    numbers=left,
}




\usepackage{fancyhdr}
\pagestyle{fancy}
\lhead{}
\chead{}
\rhead{COMP 360: Matt Adelman, Evan Carmi, Adam Forbes}
\lfoot{}
\cfoot{\thepage}
\rfoot{}

\newcommand{\HRule}{\rule{\linewidth}{0.5mm}}

\newcommand{\ty}[1]{\texttt{#1}}

\begin{document}

\begin{titlepage}

\begin{center}
% Upper part of the page
\textsc{\LARGE COMP 360}\\[1.5cm]
\textsc{\Large Final Project}\\[0.5cm]
% Title
\HRule \\[0.4cm]
{ \huge \bfseries RSA Moduli Pitfalls}\\[0.4cm]
\HRule \\[1.5cm]
% Author and supervisor
\begin{minipage}{0.4\textwidth}
\begin{center} \large
\emph{Authors:}\\
Matt Adelman\\Evan Carmi\\Adam Forbes
\end{center}
\end{minipage}
\vfill
% Bottom of the page
{\large \today}
\end{center}
\end{titlepage}

\tableofcontents

\newpage

\begin{abstract}
  Hello world, this is the abstract.  Hello world, this is the abstract.
  Hello world, this is the abstract.  Hello world, this is the abstract.  Hello world, this is the abstract.  Hello world, this is the abstract.  Hello world, this is the abstract.  Hello world, this is the abstract.  Hello world, this is the abstract.
\end{abstract}

%\begin{multicols}{2}
\section{History:}
RSA is a public key encryption algorithm developed by Ron Rivest, Adi Shamir, 
and Leonard Adleman at MIT in 1977. Rivest and Shamir are both computer
scientists that were working on an ``unbreakable'' public key encryption method.
Rivest and Shamir worked on many different codes, and would pose them as a
challenge to Adleman. Forty two of these codes were presented, and Adleman broke
them all. Finally on attempty number 43, they created what is now known as the
RSA scheme. Incidentally, English Professor CLifford Cocks developed the exact
same encryption system in 1973, but it was classified as top-secret, so it was
not released until 1997. The RSA algorithm was released for public domain in
1997.\\
The algorithm operates by using two distinct, large prime numbers to generate
public and private keys. Anyone can use the public key to encrypt a message, but
only someone with the private key can decrypt it. The algorithm is hard to
break, because if the prime numbers are large enough, the factorization is
exponential in time.

\section{Original Ideas:}

\section{Lit Review:}

\section{Algorithm Implementation:}
For our implementation, we chose JavaScript as our environment because of its
ease for web development. We really wanted to end up with something that we
could show off and display in a public setting. The code to the implementation
of our RSA algorithm can be found in Appendix A. For our implementation, we
relied heavily on the BigInt library found here: 
http://leemon.com/crypto/BigInt.js.
While there were other BigInt libraries, out there but this is the one that had
all of the functions we needed, and was implemented in an easy format for us to
deal with. The main functions we used from this library were \ty{str2bigInt}
which takes a string and a base that represents the big integer and converts it
for us. We also used \ty{inverseMod} which takes a big integer and a modulus as
a big integer, and returns the big integer that is the inverse if one exists,
otherwise \ty{null}. We also used \ty{mult} which has the obvious function of
multiplying two big integers. We also have \ty{greater} which returns a boolean
depending on whether or not the first argument is greater than the second. We
also used \ty{modPow(a,b,c)} which takes three big integers and returns $a^b
\mod c$.  
The last function we used from Leemon was \ty{addInt} which takes a big integer
and an integer and adds the integer to the big integer. This was used in the
computation of $\phi(n)$. The functions we created are as follows:
\ty{generate\_key(p:String, q:String, e:String) = [[n:BigInt, e:BigInt],
[n:BigInt, d:BigInt]]} where $p$ and $q$ are our distinct primes and $e$ is the
public exponent all represented as strings. We then return $n$ which is the
modulus, with $n = pq$. Together with $e$ the public exponent and $d$ the
private exponent. These together make up our keys. The second function we
created is \ty{crypt(key:[BigInt,BigInt], message:BigInt) = message':BigInt}
where the $key$ is a modulus with an exponent, and the $message$ is something to
be encrypted or decrypted depending on what $key$ with which we call \ty{crypt}.

The next step for our implementation was getting everything to play nice with
our web implementation (found here: http://carmi.github.com/rsa/code/ ). The 
method was to create some helper methods that would access the data from our 
form, check for errors on input and
even randomly generate pseudo-random primes based on a bit length. The code for
these helper functions can be found in Appendix A. Some of the big ones are the
\ty{gen} function which takes the information given in the form, tests for
errors such as bad public exponents, or if the public exponent is not relatively
prime to $\phi(n)$. We also have functions to encrypt and decrypt messages given
that we have public and private keys entered into our form.

We also wrote a testing suite for our JavaScript implementation to make sure it
was sane. This included minor tests for generation, encryption, decryption, and
even a few number theory functions which were in our original implementation of
the RSA algorithm that only worked on integers. This is actually when we
realized we needed to switch to arbitrary precision integers, because the
exponents get very large, very quickly.






To ensure that future changes do not introduce bugs into our already working codebase we use tests. For our JavaScript implementation of RSA we used QUnit tests, a testing framework in JavaScript used by jQuery among others. This allows our tests to be run easily in the browser along side our implementation. If future changes break a function, then the associated test will fail alerting us to the portion of code that is broken.




\section{Data Collection:}
Any analysis of RSA certificates requires a set of data with which to test. For
our project we needed a large set of data so as to best mimic the data that
Lenstra et al used in their analysis of public keys. Lenstra et al stated in their paper
that they used the Electronic Frontier Foundation's SSL Observatory -
https://www.eff.org/observatory. We initially looked at the SSL Observatory as a
possibly source of data for our project. However, unfortunately the SSL
Observatory has not been updated in a few years. Furthermore, it includes no
complete guide on how to use its source, and is simply a collection of python
scripts that interact with a \ty{MySQL} database in distinct ways. Thus, we were
unable to successfully use the SSL Observatory source code as a means to gather
RSA public keys. However, there exists data dumps from the SSL Observatory from
2010. We gathered this data, available through \ty{bittorrent}, and used it as a
starting point.

In order to see the effectiveness and change since the Lenstra paper was
published in early 2012 we needed to scrap for fresh data. (The paper briefly
mentions that the latest data they looked at was from February 2012.) The SSL
Observatory scans all \ty{IPv4} space for servers that respond to https requests. We
did not have the computational time/power to do this, so we decided on the Alexa
top 1 million domains as a source of hosts. Then for each one of these domains
we attempted to connect via HTTPS and collect the public key used for the
connection. The specific method of collecting these moduli was the  This was
done by \ty{curl}'ing the domain at port 443 with a \ty{HTTP} request. If we received a
response, we used \ty{openssl s\_client} to get the decoded certificate and
\ty{openssl x509} to decode the certificate into a easily passable format.
However, we noticed that less than 10\% of domains responded to our requests. We
added some further error handling for these situations. We followed redirects
and tried a few common subdomains to improve our rate of responses to over 12\%.
Out of the 1000000 domains we received 125723 X.509 certificates each containing
a moduli. This was the source of our data for our experiments.

\section{Experiments:}
The main idea of our experiments is as follows: if we have two moduli $m, n$ we
can find if these two moduli share a prime $p$ simply by taking the $\gcd$ of
the two moduli. This is very fast and is linear in the number of bits. To do
around 100000 $\gcd$ operations on the machines we were using took around 5
seconds each. Therefore our code, seen in Appendix A, does the following: It
takes a file that contains the list of moduli, along with other identifying 
information. It also takes a smaller file which only contains a small section of
the moduli. This was done so the processing could be broken up onto multiple
machines. Once the two files are read into memory, the program stores the moduli
in an array, and a unique identifier corresponding to the moduli in another
array. We then loop through the data comparing every modulus in the smaller file
with every modulus in the bigger file.

\section{Analysis of Data:}

\section{Problems:}

\section{Conclusion:}

\section{Further Research:}

%\end{multicols}

\appendix
\section{Appendix of code:}



\lstinputlisting[caption=A ruby script that interfaces with unix command line utilities to collect X.509 certificates given a list of hosts.]{ssl_cert_scraper.rb}


\end{document}

Abstract (All):

History (Matt):

Lit Review (Adam):
 - RWWWIR

Original ideas:

Algorithm Implementaion (Matt):
 - What we did
 - Big int package used
 - Why web implementaion

Data collection (Evan):
- SSL observatoru 2010
- new data collection
- parsig new data
- analysing new data (and old data)

analysis

conclusion, things to do better

further research/work
   - database of moduli that you can check on

- code



Possible tables and graphs
- top ten shared moduli
- number of new data collected
    - distribution of key size
-Time complexity of expected vs. actual



% ------
% Fonts and typesetting settings
\usepackage[sc]{mathpazo}
\usepackage[T1]{fontenc}
\linespread{1.05} % Palatino needs more space between lines
\usepackage{microtype}


% ------
% Page layout
\usepackage[hmarginratio=1:1,top=32mm,columnsep=30pt,margin=1in]{geometry}
\usepackage[font=it]{caption}
\usepackage{paralist}
\usepackage{multicol}

% ------
% Lettrines
\usepackage{lettrine}


% ------
% Abstract
\usepackage{abstract}
	\renewcommand{\abstractnamefont}{\normalfont\bfseries}
	\renewcommand{\abstracttextfont}{\normalfont\small\itshape}

